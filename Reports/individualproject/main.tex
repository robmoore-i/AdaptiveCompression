\documentclass[a4paper, twoside]{report}

%% Language and font encodings
\usepackage[english]{babel}
\usepackage[utf8x]{inputenc}
\usepackage[T1]{fontenc}

%% Sets page size and margins
\usepackage[a4paper,top=3cm,bottom=2cm,left=3cm,right=3cm,marginparwidth=1.75cm]{geometry}

%% Useful packages
\usepackage{amsmath}
\usepackage{graphicx}
\usepackage[colorinlistoftodos]{todonotes}
%\usepackage[colorlinks=true, allcolors=blue]{hyperref}
\usepackage{url}
\usepackage{tcolorbox}
\usepackage{float}
\usepackage{setspace}

\title{Adaptive Compression for Graph Processing}
\author{Rob Moore}
% Update supervisor and other title stuff in title/title.tex

\begin{document}
\input{title/title.tex}

\onehalfspacing
%\doublespacing
\linespread{1.15}

\begin{abstract}
Many graph-modeled data sources are subject to fast-changing, unpredictable workloads, such as a news story being searched for as it happens, or a trending post or person on social media causing an influx of related, similar queries. Most processing frameworks for graph data sources contain preprocessing steps to speed up the overall analysis, which prohibits them from responding to queries until that step is done. In this report we explore potential solutions to this problem by building novel variations of database cracking, an algorithm designed for auto-tuning relational databases, and showing how the cracking algorithm can be changed to improve performance for graph data processing.
\end{abstract}

\renewcommand{\abstractname}{Acknowledgements}
\begin{abstract}
I would like to thank my supervisor Holger Pirk for his knowledge and direction during the project. I am also grateful to my personal tutor, Paul Kelly, for his advice about projects in general and about optimization and graph algorithms, as well as all of the interesting discussions we've had while I've been at Imperial. Finally I'd like to thank my dad who scanned the boundary pieces of this report and gave me some advice on how to improve it.
\end{abstract}

\tableofcontents
\listoffigures
\listoftables

\chapter{Introduction}

\section{Ad-hoc graph Processing}

Many popular graph processing frameworks currently used are all described in reference to an
offline graph processing situation, in which the graph has been built and later it is time to do some
analysis. In this project we propose a graph processing framework designed specifically to perform
ad-hoc graph processing with no preprocessing overhead and adaptive storage. Analysing a graph in real
time as it grows provides the potential to find and exploit insights where it would be too late if
they were found after-the-fact during offline analysis.

Our technique is designed for use with traversal algorithms - algorithms which traverse the graph
by following edges in the course of their computation.

\section{Clustering and Compression}

This project uses database cracking for adaptive clustering. We trial the use of some different
techniques for applying adaptive compression while cracking in order to achieve further performance benefits.

\section{Report Outline}

In the next chapter we will discuss related work in the area of graph processing frameworks and
workload aware frameworks, as well as graph indexing techniques.

After that there will be a chapter explaining and describing the background material required in order
to get the most out of reading this report.

The main body of the work will lie mainly in the chapter on adaptive clustering and compression, which
will include a description of the algorithms we trialled to find the most effective way of performing
adaptive graph queries.

Then there will be chapter on how we evaluated our work, beginning with a description of how the
evaluations were done, including the experimental setup, the algorithms and the datasets. The chapter
concludes with a discussion of our results.

Finally we have a conclusion chapter in which we summarise our contributions and discuss potential future work.

\section{Contributions}


\chapter{Related Work}

\label{ch:relatedwork}

\section{Workload-Aware Frameworks}

\subsection{Database Cracking}

From Idreos et al., Cracking is a relational database auto-tuning technique which performs online
restructuring of a relational table into disjoint pieces, storing information about each piece within
a separate data-structure called the cracker index.

When a column is queried, it is copied into a version of the column called the cracker column. The
cracker column is then scanned, restructuring it in-place to position the retrieved elements in
contigious positions. The indices of the bounds of the newly formed contigious region are stored
within the cracker index to inform future queries of whether they need to scan that piece of the
column.

Figure \ref{fig:cracking_img} shows two queries being run against a column within a system employing
database cracking. We can see that Q1 partitions copies the column into a cracker column, which is
then partitioned into pieces, of which information is known about the contents. Q2 further breaks
up the column into pieces, and again the information about the newly formed partitions is stored
in the cracker index.

\begin{figure}[h]
  \includegraphics[width=\textwidth]{cracking_img}
  \caption{Illustration of database cracking}
  \label{fig:cracking_img}
\end{figure}

\subsection{Group-by-Query}

In-part inspired by cracking was the thesis work of Aluç, who proposed a group-by-query (G-by-Q)
representation for RDF data, for which the structure of individual database records, as well as
the way records are serialized on the storage system are dynamically determined based on the
workload. This technique proved to be fast and robust against other popular frameworks for
querying RDF data, however, the system is complicated - Aluç's implementation was reported to be
over 35,000 lines of C++. In this work we are aiming to produce simpler contributions.

\section{Graph Processing Frameworks}

\subsection{Ligra Framework}

Ligra is a lightweight graph processing framework for shared-memory multi-core machines for graph
traversal algorithms, such as pagerank and BFS. They were in part inspired by Beamer et al.'s work
with shared-memory machines, acheiving speed-up by dynamically switching between push and pull implementations of BFS based on the graph's density. Ligra takes the form of a simple API of three
routines: size, edgeMap and vertexMap.

size takes a set of vertices and returns the number of them.

edgeMap applies a function F to all out-neighbours of a vertex U which satisfy the condition C.

vertexMap applies a function F to every element in a set of vertices.

\section{Graph Indices}

\subsection{Space filling curve layouts}

The most famous example of this is the Hilbert Curve, which is a continuous mapping between a number
and a point on a 2D square. Numbers which are closeby are mapped closely together - in this way the
hilbert ordering of an index of an adjacency list is somewhat locality preserving. We can use a
hilbert ordering in order to improve locality on edge accesses, which therefore leads to improved
cache performance.

Another example of using a space filling curve to preprocess graph data is with co-ordinate 

\subsection{Frequency Based Clustering}

Frequency based clustering constitutes physically reorganising the vertex data such that frequently
accessed vertices are clustered. This improves cache utilisation and reduces the cycles spent stalled
on memory.

\chapter{Background}

\label{ch:background}

\section{Database Cracking}

In this section we'll give a more detailed look at the cracking algorithm and the way in which we
have employed it in this project.

The algorithm we describe here as "the database cracking algorithm" is actually the "crack in three"
variant of the algorithm as described by Idreos. This algorithm is a scan of a partition of a column in which values are rearranged in order to achieve a partial sort on the column. Through repeated
scans of this nature the column approaches and eventually reaches a fully sorted state. There is also
the possibility of not fully sorting the column, instead scanning naively provided the fragment
under inspection is below a threshold size.

In the first stage of the algorithm, a contiguous section of the column is selected for scanning by
selecting two pointers from the cracker index, between which all of the result tuples are known to
lie. If the cracker index is empty, then the entire column is selected as the sole fragment of the
column. We will commonly refer to a section of the column which can be specified by values inside the cracker index as column fragments, or just fragments.

If \texttt{(value, index)} is stored in the cracker index, then at all indices preceeding
\texttt{index}, the corresponding value in the cracker column is strictly less than \texttt{value}.

In order to map back from the cracker column to the original columns of the table, we use an array
which is initialised to store successive integers from 0, of the same length as the table. We apply
all swapping operations to this array as well as the cracker column, which means that any range of
values in the cracker column have indices which map to those same values in the original column of
the original table. We refer to this array as the \texttt{base\_idx} array

We define two predicates with respect to the selected range of values. The first returns true when
its input is outside of the selection on the lower end, the second returns true if the input is
outside the selection on the higher end. We will refer to these predicates as lowp and highp
respectively.

The selected fragment is defined by two edge pointers which we have looked up. One pointing at the
lower side of the fragment, \texttt{low\_ptr}, and one pointing at the upper side, \texttt{high\_ptr}.
At all times during the algorithm, we will maintain the invariant that all indices before
\texttt{low\_ptr} will point to a value in the cracker column which satisfies lowp. Similarly, all
pointers after \texttt{high\_ptr} point to a value in the cracker column satisfying highp. By "all pointers", I mean all valid pointers across the entire of the cracker column.

We then tighten the two edge pointes inwards while the associated invariant holds. We will refer to
this procedure as tightening.

The next part involves scanning the fragment between the edge pointers. An iteration pointer,
\texttt{itr\_ptr}, begins at the same point as \texttt{low\_ptr}, scanning upwards. For each value that it encounters in the cracker column, it is determined using the predicates, in which region of
the column the value belongs - the low side out-of-bounds region, the selected region, or the upper
side out-of-bounds region. If the value is not selected, that is, if it's out-of-bounds, it is swapped
with the value at the respective edge pointer. The edge pointer is then tightened - resulting in it
moving inwards by at least one, since the swapped-in value was determined to be out-of-bounds on that
side. During the scan, we maintain the loop invariant that all indices before \texttt{itr\_ptr} point
to a value in the cracker column which satisfies either lowp, or is within the selected range, which
is equivalent to saying that all the indices from \texttt{low\_ptr} inclusive to \texttt{itr\_ptr}
exclusive are within the selected range. At the end of the algorithm therefore, we will find that
this corresponds with the inclusive range from \texttt{low\_ptr} to \texttt{high\_ptr}, which is
defined by lowp values earlier in the column and highp values later in the column.

After the scan is concluded, we have access to some information about the column we wish to store
within the cracker index. The value at \texttt{low\_ptr} in the cracker column exceeds the values in all preceeding indices, therefore we store the key-value pair (cracker-column[\texttt{low\_ptr}],
\texttt{low\_ptr}) to indicate this fact. Similarly, we know that all values at indices strictly greater than \texttt{high\_ptr} are strictly greater than the value stored at \texttt{high\_ptr}, so
we store (cracker-column[\texttt{high\_ptr}] + 1, \texttt{high\_ptr} + 1).\footnote{For integers, the
minimum difference between elements is 1, however, for other datatypes, this may be different. See
the subsection on per-fragment compression within the adaptive compression chapter for a brief discussion on this point.}

Having acquired the range of indices in which our desired range of values lies in the cracker column,
we map this back to indices of the original columns using the \texttt{base\_idx} array, collect the
values from the required columns, and return them.

\input{project/project.tex}
\chapter{Evaluation}

\label{ch:evaluation}

\section{Experimental Setup}

The experiments were performed on the department of computing lab machines, $sprite30$ and $sprite29$. These machines have a "HP EliteDesk 800 G2 TWR, Intel Core i7-6700 3.40GHz" CPU and have 16GB of RAM.

\section{Datasets}

\subsection{LDBC Social Network Benchmark Data Generator}

To test our work, we have used the LDBC-SNB DataGen program \cite{Erling:2015:LSN:2723372.2742786}, which generates a social network graph to a number of scale factors. We used scale factors 1, 3 and 10. We would have used larger data but for the size of the machine we performed our experiments with. The number of nodes and edges in these graphs is shown in table \ref{tab:datasets_table}.

\begin{table}[H]
    \centering
    \begin{tabular}{| l | l | l |}
    \hline
    Scale Factor & Nodes & Edges \\ \hline
    1 & 9893 & 180624 \\ \hline
    3 & 24329 & 565248 \\ \hline
    10 & 65646 & 1938517 \\
    \hline
    \end{tabular}
    \caption{LDBC-SNB generated datasets}
    \label{tab:datasets_table}
\end{table}

A major use-case for adaptive indexing on graph data is the prospect of finding trends within a social network. When something happens in the news, for example, the workload changes, requiring a new index to improve performance. We have used a simple popularity/trend-analysis measure in pagerank ("personrank" in our case) to assess the performance of our system.

In this network, people are nodes and their relationships are edges. The node ids are unchanged from the data generation, they are not mapped onto consecutive integers from 0.

\subsection{Randomly Generated Trees}

We wished to find the break-even point of up-front sorting versus our implementations, so to support this, we used breadth-first search on randomly generated trees. To find the "break-even point", we find the time it takes to sort the edge array using quicksort \cite{doi:10.1093/comjnl/5.1.10} and then count the number of queries the cracking implementation has already answered in that time. This is used to determine the gains we can make by using adaptive indexing versus using offline index creation, that is, up-front sorting.

\section{Graph Algorithms}

\subsection{Breadth First Search}

Breadth first search, or BFS, involves choosing a starting node as the sole member of a frontier, which then expands in iterations in which members of the frontier append their out-neighbours which have not yet been visited to the frontier, and remove themselves. This continues until all vertices have been visited.

The most important factor for us in considering BFS is that it queries the outgoing edges of each node just once, meaning that compressed column fragments are never revisited during the course of the algorithm. For this reason we believe it provides an appropriate lower bound on the break-even point.

\subsection{Pagerank}

Pagerank \cite{ilprints422} is a famous algorithm which stores a rank for each vertex and iteratively updates the values across the entire graph. All nodes are initialized with a rank of $\frac{1}{|V|}$. During each iteration, each node inherits from all of its in-neighbours a contribution of their pagerank divided by their out-degree. This value is then multiplied by a damping factor and then added to a base value to give the updated rank. The base value is defined as $\frac{1 - d}{|V|}$.

In pagerank, every iteration considers all of the vertices and edges in the graph, and so over an execution of many iterations, any clustering or compression will see further benefits compared to BFS.

To benchmark our contributions, we have implemented pagerank as an equivalent "personrank" for the generated social network.

\section{Systems under Comparison}

The purpose of this project is not to create the fastest system, but to assess the applicability towards graph processing of compression-based variants of cracking. To ensure that our experiments fairly reflect this, we have chosen to compare our single-threaded cracking implementation to other single-threaded solutions. Predicated and vectorized implementations of cracking have been done and been shown to be highly effective for improving the CPU efficiency of cracking against the original algorithm \cite{Pirk:2014:DCF:2619228.2619232}, however, we have not applied these improvements to our algorithm. This is not a problem because our work is concerned primarily with the fundamental differences between standard cracking and cracking with adaptive compression, rather than optimizing them especially, although we touch on potential optimizations in our discussion of future work.

With that in mind, we have tested our system against the standard cracking algorithm and against upfront quicksort. The reason we have chosen quicksort specifically is for its similarity to cracking, in that it uses pivots to partially sort column sections. Compare this to cracking, which uses pivots to partition the column towards being fully sorted.

Our contributions featured variations in two dimensions: Opportunities to compress and storage of compression information. We have implemented per-fragment compaction, per-fragment recognition, underswapping RLE recognition and overswapping RLE recognition. The reason we decided not to implement compaction with RLE compression is that it seemed clear from the disappointing results of per-fragment compaction that it would not be worth it.

\section{Results}

\subsection{Personrank}

Using data generated by the LDBC-SNB at scale factors of 1, 3 and 10, we measured the wall-clock time to completion of pagerank implementations using each adaptive indexing technique. We have also included the results for an implementation using upfront quicksort.

For each of the three graphs, we ran pagerank over 50 and 100 iterations for every method. Each result is averaged over 10 runs. Blue and red represent the results over 50 and 100 iterations respectively.

\begin{figure}[H]
  \centering
  \includegraphics[width=\textwidth]{images/personrankSF1}
  \caption{Personrank execution times for the SF1 social network graph}
  \label{fig:personranksf1}
\end{figure}

\begin{figure}[H]
  \centering
  \includegraphics[width=\textwidth]{images/personrankSF3}
  \caption{Personrank execution times for the SF3 social network graph}
  \label{fig:personranksf3}
\end{figure}

\begin{figure}[H]
  \centering
  \includegraphics[width=\textwidth]{images/personrankSF10}
  \caption{Personrank execution times for the SF10 social network graph}
  \label{fig:personranksf10}
\end{figure}

Our experiments showed that our implementation of compaction is always the slowest, regardless of data size or the number of iterations. This is due to the prohibitively high costs associated with the constant copying of memory which this technique depends on.

We also showed however, that per-fragment recognition was always the fastest. This is because, by recognizing a uniform column-fragment, we can avoid doing any scanning at all. This optimization to the original algorithm yielded a small average speedup across our six reported executions of 8.1$\%$.

The two RLE methods we developed showed promise. Across our reported executions, they were faster than standard cracking as often as they were slower. We found very little difference in performance between underswapping and overswapping, although this is likely to be because they are inherently very similar, differing only in the way they perform high-side swaps of two runs of different lengths - a case which is evidently not encountered enough to show any sizable performance difference between the two different implementations of that case.

\subsection{Break-even Point}

Using randomly generated trees, we ran BFS using each adaptive method and counted how many cracking queries were completed in the same amount of time as it took to sort the tree using quicksort.

The reason we chose to do BFS on trees, is that it represents a case in which every query is guaranteed to be doing a scan and not returning compressed values. We did this in order to assess how much of an impact on the break-even performance and early-scanning speed our changes had to the original cracking algorithm.

We evaluated the break-even point for trees of size 100,000, 500,000 and 1,000,000 (number of nodes), averaged over 10 runs, with a new tree for each run. Our results are shown in figure \ref{fig:breakeven}. Blue, red and yellow represent the results for the trees with 100,000, 500,000 and 1,000,000 nodes respectively.

\begin{figure}[h]
  \centering
  \includegraphics[width=\textwidth]{images/breakeven}
  \caption{Break-even point for each studied adaptive technique}
  \label{fig:breakeven}
\end{figure}

Compaction performed poorly compared to standard cracking in this test. We see that with greater data size, the number of queries ran by compaction in the same time as it took to sort the array decreases. This makes sense given that the primary cost factor in compaction is the copying of memory. As the size of the array increases and the number of copied values increases, so the performance during early queries will decrease.

Per-fragment recognition makes no changes to the cracking scan, and so it makes sense that there would be little difference in the break-even point compared to standard cracking, regardless of the data size.

The two RLE cracking variants actually change the implementation of the scan - they perform extra work to build and maintain knowledge about runs in exchange for the ability to exploit this knowledge. In the early queries we expect the costs to outweigh the benefits, because at the start, there is no run-length information to exploit, but at the same time there is a lot of new information to build and write into memory, which is expensive. This results in a significantly higher relative cost for early queries when using RLE cracking, causing the break-even point to be dramatically lower. This is unfortunate, because one of the main advantages of using cracking over a non-adaptive indexing technique is that it is lightweight - a property lost by performing expensive run-length encoding in early queries.

Compared to overswapping, underswapping performs less work during high-side swapping. It has fewer branches because it does not have to deal with padding or overlapping. The earliest queries perform the largest scans and therefore perform the most swaps. It therefore makes sense that the relative advantage underswapping has in performing less work during swapping would be most evident in the earliest queries.

\subsection{Summary}

\begin{itemize}
\item \textbf{Per-fragment Compaction}: This method was quite disappointing, although it cannot be said that it was unexpected. The costs of deleting arbitrary elements from an array and copying the non-contiguous data towards the head are too high to make this a viable technique.
\item \textbf{Per-fragment Recognition}: By recognizing uniform column pieces using the invariants of the cracker index, we can prevent unnecessary scans, which improves performance. We showed that per-fragment recognition is an effective way to improve the performance of cracking as an adaptive index on an edge-array representing a graph.
\item \textbf{RLE}: We showed that run-length encoding an edge array during cracking is a technique that performs similarly to standard cracking in terms of overall speed, but is a far less lightweight operation.
\end{itemize}
\chapter{Conclusion}

\label{ch:conclusion}

\section{What we did}

We implemented four compression-based variations of the original cracking algorithm and applied them to graph data. We evaluated our implementations using pagerank and by finding the break-even point point for each algorithm, comparing their performance to that of standard cracking and to upfront sorting. 

From the evaluation of these methods, we found compaction to be unsuitable due to the prohibitively high cost of compacting values. We found run-length encoding to perform similarly to standard cracking for overall speed in the execution of pagerank, however we found that the costs added to the scanning phase cause early queries to not retain the lightweight property of standard cracking. This caused its break-even point against upfront sorting to be far lower. Finally, we found a positive result among our explored variations - recognition-based compression was shown to be an improvement over standard cracking for our pagerank experiments, with the highest recorded cost to the break-even point being less than 2$\%$.

\section{Future Work}

\begin{itemize}
\item \textbf{CPU efficient implementations}: None of our implementations used predication or vectorization in order to improve their CPU efficiency, which is known to be effective\cite{Pirk:2014:DCF:2619228.2619232}. Standard cracking sees a significant speed up in single-threaded performance for the predicated and vectorized implementation. It would be more complicated for the run-length encoding variant of cracking, because the implementation for standard cracking is fairly dependent on processing a single value at a time. This is the case for the backup and active slots for the predicated backup technique and also in accounting for buffer overflow in the vectorized implementation.
\item \textbf{Parallel implementations}: We studied only single-threaded implementations of our algorithms. Parallel implementations of them present interesting difficulties, because the refined partition \& merge algorithm\cite{Pirk:2014:DCF:2619228.2619232} could not be naively copied over to a run-length encoding form of cracking - we would need a way to initialize the multiple edge pointers used such that they did not lie within runs. Furthermore, when swapping runs of different lengths, we would frequently encounter situations in which the high-side block of a separated block within the refined partition \& merge could not hold the different sized run due to overlapping with another block. These difficulties might suggest that we would be better off using the simple partition \& merge, despite its poorer performance, however that is all a possibility for future work.
\item \textbf{Stochastic RLE Cracking}\cite{Halim:2012:SDC:2168651.2168652}: One of the properties of stochastic cracking that makes it an appealing choice for application towards RLE is that it improves the workload-robustness of it. In that, it aims to prevent unnecessary physical reorganization and scanning - a property that RLE cracking would benefit greatly from, given that the cost of scanning and swapping in RLE cracking is what makes it so much less lightweight that standard cracking.
\item \textbf{Different compression schemes}: We used run-length encoding, but we didn't look at any other compression formats, such as compression formats optimized for different types of graph queries, such as point queries, which aim to find matching subgraphs within a database.
\end{itemize}
\input{appendix/appendix.tex}

\renewcommand\bibname{References}
\bibliographystyle{abbrv}
\bibliography{bibs/refs}

\end{document}